\documentclass[12pt,a4paper,portrait]{article}
%\setcounter{secnumdepth}{0}
\usepackage{gensymb}
\usepackage{pdflscape}
\usepackage{amsmath}
\usepackage{amssymb}
\usepackage{enumitem}
\usepackage{graphicx}
\usepackage{subcaption}
\usepackage{multirow}
\usepackage{sansmath}
\usepackage{pst-eucl}
\usepackage{multicol}
\usepackage{csquotes}
% Coding
\usepackage{listings}
\setlength{\parindent}{0pt}
\usepackage[obeyspaces]{url}
% Better inline directory listings
\usepackage{xcolor}
\definecolor{light-gray}{gray}{0.95}
\newcommand{\code}[1]{\colorbox{light-gray}{\texttt{#1}}}
\usepackage{adjustbox}
\usepackage[UKenglish]{isodate}
\usepackage[UKenglish]{babel}
\usepackage{float}
\usepackage[T1]{fontenc}
\usepackage{setspace}
\usepackage{sectsty}
\usepackage{longtable}
\newenvironment{tightcenter}{%
	\setlength\topsep{0pt}
	\setlength\parskip{0pt}
	\begin{center}
	}{%
	\end{center}
}
\captionsetup{width=\textwidth}
\usepackage{mbenotes} % to print table notes!
\usepackage{alphalph} % For extended counters!
% usage: \tabnotemark[3]\cmsp\tabnotemark[4]
\usepackage[colorlinks=true,linkcolor=blue,urlcolor=black,bookmarksopen=true]{hyperref}
\sectionfont{%			            % Change font of \section command
	\usefont{OT1}{phv}{b}{n}%		% bch-b-n: CharterBT-Bold font
	\sectionrule{0pt}{0pt}{-5pt}{3pt}}
\subsectionfont{
	\usefont{OT1}{phv}{b}{n}}
\newcommand{\MyName}[1]{ % Name
	\usefont{OT1}{phv}{b}{n} \begin{center}of {\LARGE  #1}\end{center}
	\par \normalsize \normalfont}
\makeatletter
\newcommand\FirstWord[1]{\@firstword#1 \@nil}%
\newcommand\@firstword{}%
\newcommand\@removecomma{}%
\def\@firstword#1 #2\@nil{\@removecomma#1,\@nil}%
\def\@removecomma#1,#2\@nil{#1}
\makeatother

\newcommand{\MyTitle}[1]{ % Name
	\Huge \usefont{OT1}{phv}{b}{n} \begin{center}#1\end{center}
	\par \normalsize \normalfont}
\newcommand{\NewPart}[1]{\section*{\uppercase{#1}}}
\newcommand{\NewSubPart}[1]{\subsection*{\hspace{0.2cm}#1}}
\renewcommand{\baselinestretch}{1.05}
\usepackage[margin=0.2cm]{geometry}
\date{}
\title{Elastic pendulum}
\author{Brenton Horne}

\begin{document}
\maketitle

Say we have a mass $m$ attached to a spring of rest length $l_0$. Suppose we call the displacement from rest $z$. That way the length of the spring is $l(t) = l_0 + z$. If we measure $\theta$ clockwise from the positive x-axis, then our Cartesian coordinates are defined as:

\begin{align*}
	x &= (l_0+z)\cos{\theta} &\implies \dot{x} &= \dot{z}\cos{\theta} - (l_0+z)\dot{\theta}\sin{\theta}\\
	y &= (l_0+z)\sin{\theta} &\implies \dot{y} &= \dot{z}\sin{\theta} + (l_0+z)\dot{\theta}\cos{\theta}.\\
\end{align*}

Therefore:
\begin{align*}
    v^2 &= \dot{x}^2+\dot{y}^2 \\
	&= \left[\dot{z}\cos{\theta} - (l_0+z)\dot{\theta}\sin{\theta}\right]^2 + \left[\dot{z}\sin{\theta} + (l_0+z)\dot{\theta}\cos{\theta}\right]^2 \\
	&= \dot{z}^2 \cos^2{\theta} + (l_0+z)^2\dot{\theta}^2\sin^2{\theta} - 2\dot{z}\dot{\theta}(l_0+z)\cos{\theta}\sin{\theta} + \dot{z}^2\sin^2{\theta} + (l_0+z)^2\dot{\theta}^2\cos^2{\theta} + 2\dot{z}\dot{\theta}(l_0+z)\sin{\theta}\cos{\theta} \\
	&= \dot{z}^2 + (l_0+z)^2\dot{\theta}^2.
\end{align*}

Hence the kinetic energy is:

\begin{align*}
	T &= \dfrac{m}{2} v^2 \\
	&= \dfrac{m}{2} \left[\dot{z}^2 + (l_0+z)^2\dot{\theta}^2\right].
\end{align*}

As for the potential energy, it will have two components: a spring component,

\begin{align*}
	V_S &= \dfrac{kz^2}{2};
\end{align*}

and a graphical component,

\begin{align*}
	V_G &= mgy \\
	&= mg(l_0+z)\sin{\theta}.
\end{align*}

Therefore the total potential energy is:

\begin{align*}
	V &= V_S + V_G \\
	&= \dfrac{kz^2}{2} + mg(l_0+z)\sin{\theta}.
\end{align*}

Hence the Lagrangian is:

\begin{align*}
    \mathcal{L} &= T - V \\
    &= \dfrac{m}{2} \left[\dot{z}^2 + \dot{\theta}^2(l_0+z)^2\right] - \dfrac{kz^2}{2} - mg(l_0+z)\sin{\theta}.
\end{align*}
\end{document}