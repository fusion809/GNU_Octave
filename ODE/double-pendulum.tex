\documentclass[12pt,a4paper,portrait]{article}
%\setcounter{secnumdepth}{0}
\usepackage{gensymb}
\usepackage{pdflscape}
\usepackage{amsmath}
\usepackage{amssymb}
\usepackage{enumitem}
\usepackage{graphicx}
\usepackage{subcaption}
\usepackage{multirow}
\usepackage{sansmath}
\usepackage{pst-eucl}
\usepackage{multicol}
\usepackage{csquotes}
% Coding
\usepackage{listings}
\setlength{\parindent}{0pt}
\usepackage[obeyspaces]{url}
% Better inline directory listings
\usepackage{xcolor}
\definecolor{light-gray}{gray}{0.95}
\newcommand{\code}[1]{\colorbox{light-gray}{\texttt{#1}}}
\usepackage{adjustbox}
\usepackage[UKenglish]{isodate}
\usepackage[UKenglish]{babel}
\usepackage{float}
\usepackage[T1]{fontenc}
\usepackage{setspace}
\usepackage{sectsty}
\usepackage{longtable}
\newenvironment{tightcenter}{%
	\setlength\topsep{0pt}
	\setlength\parskip{0pt}
	\begin{center}
	}{%
	\end{center}
}
\captionsetup{width=\textwidth}
\usepackage{mbenotes} % to print table notes!
\usepackage{alphalph} % For extended counters!
% usage: \tabnotemark[3]\cmsp\tabnotemark[4]
\usepackage[colorlinks=true,linkcolor=blue,urlcolor=black,bookmarksopen=true]{hyperref}
\sectionfont{%			            % Change font of \section command
	\usefont{OT1}{phv}{b}{n}%		% bch-b-n: CharterBT-Bold font
	\sectionrule{0pt}{0pt}{-5pt}{3pt}}
\subsectionfont{
	\usefont{OT1}{phv}{b}{n}}
\newcommand{\MyName}[1]{ % Name
	\usefont{OT1}{phv}{b}{n} \begin{center}of {\LARGE  #1}\end{center}
	\par \normalsize \normalfont}
\makeatletter
\newcommand\FirstWord[1]{\@firstword#1 \@nil}%
\newcommand\@firstword{}%
\newcommand\@removecomma{}%
\def\@firstword#1 #2\@nil{\@removecomma#1,\@nil}%
\def\@removecomma#1,#2\@nil{#1}
\makeatother

\newcommand{\MyTitle}[1]{ % Name
	\Huge \usefont{OT1}{phv}{b}{n} \begin{center}#1\end{center}
	\par \normalsize \normalfont}
\newcommand{\NewPart}[1]{\section*{\uppercase{#1}}}
\newcommand{\NewSubPart}[1]{\subsection*{\hspace{0.2cm}#1}}
\renewcommand{\baselinestretch}{1.05}
\usepackage[margin=0.2cm]{geometry}
\date{}
\title{Double pendulum: damping and non-zero mass of bob and rod}
\author{Brenton Horne}

\begin{document}
	\maketitle
	
	To obtain our equations of motion, we must apply the Euler-Lagrange equation with a generalized dissipation force ($Q_i = \sum_j \vec{F}_{D,j} \cdot \dfrac{\partial \vec{r_j}}{\partial q_i}$):
	
	\begin{align}
		\dfrac{d}{dt}\dfrac{\partial \mathcal{L}}{\partial \dot{q_i}} - \dfrac{\partial \mathcal{L}}{\partial q_i} &= Q_i. \label{ELD}
	\end{align}
	First we must determine the kinetic energy of the first rod:
	\begin{align*}
		T_{\mathrm{rot}, 1} &= \dfrac{1}{2} I_{\mathrm{cm}} \dot{\theta}_1^2 \\
		T_{\mathrm{rot}, 1} &= \dfrac{m_{1r} r_1^2\dot{\theta}_1^2}{12} \\
		T_{\mathrm{rot}, 1} &= \dfrac{m_{1r} r_1^2\dot{\theta}_1^2}{24}.
	\end{align*}
	
	The second rod should be the same except with its own parameters:
	\begin{align*}
		T_{\mathrm{rot}, 2} &= \dfrac{m_{2r} r_2^2\dot{\theta}_2^2}{24}.
	\end{align*}
	
	As for the mass of the first pendulum bob:
	\begin{align*}
		T_{\mathrm{bob}, 1} &= \dfrac{m_1v_1^2}{2}.
	\end{align*}
	
	The first pendulum bob would have the coordinates:
	\begin{align*}
		x_1 &= r_1 \cos{\theta_1} &\implies \dot{x}_1 &= -r_1\dot{\theta}_1 \sin{\theta_1}\\
		y_1 &= r_1 \sin{\theta_1} &\implies \dot{y}_1 &= r_1 \dot{\theta}_1 \cos{\theta_1}. \\
	\end{align*}
	
	Therefore:
	
	\begin{align*}
		v_1^2 &= \dot{x}_1^2 + \dot{y}_1^2 \\
		&= \left(-r_1\dot{\theta}_1 \sin{\theta_1}\right)^2 + \left(r_1 \dot{\theta}_1 \cos{\theta_1}\right)^2 \\
		&= r_1^2 \dot{\theta}_1^2.
	\end{align*}
	
	And the kinetic energy of the first pendulum bob is:
	
	\begin{align*}
		T_{\mathrm{bob}, 1} &= \dfrac{m_{1b} v_1^2}{2} \\
		&= \dfrac{m_{1b} r_1^2 \dot{\theta}_1^2}{2}.
	\end{align*}
	
	Second pendulum bob. The coordinates are:
	
	\begin{align*}
		x_2 &= x_1 + r_2 \cos{\theta_2} \\
		&= r_1 \cos{\theta_1} + r_2 \cos{\theta_2} &\implies \dot{x}_2 &= -r_1\dot{\theta}_1 \sin{\theta_1}-r_2\dot{\theta}_2 \sin{\theta_2}\\
		y_2 &= y_1 + r_2 \sin{\theta_2} \\
		&= r_1 \sin{\theta_1} + r_2 \sin{\theta_2} &\implies \dot{y}_2 &= r_1\dot{\theta}_1 \cos{\theta_1}+r_2\dot{\theta}_2 \cos{\theta_2}.
	\end{align*}
	
	The velocity is therefore:
	
	\begin{align*}
		v_2^2 &= \dot{x}_2^2 + \dot{y}_2^2 \\
		&= \left(-r_1\dot{\theta}_1 \sin{\theta_1}-r_2\dot{\theta}_2 \sin{\theta_2}\right)^2 + \left(r_1\dot{\theta}_1 \cos{\theta_1}+r_2\dot{\theta}_2 \cos{\theta_2}\right)^2 \\
		&= r_1^2 \dot{\theta}_1^2 \sin^2{\theta_1} + r_2^2 \dot{\theta}_2^2 \sin^2{\theta_2} + 2r_1 r_2 \dot{\theta}_1\dot{\theta}_2 \sin{\theta_1}\sin{\theta_2} + r_1^2 \dot{\theta}_1^2 \cos^2{\theta_1} + r_2^2 \dot{\theta}_2^2 \cos^2{\theta_2} + 2r_1 r_2 \dot{\theta}_1 \dot{\theta}_2 \cos{\theta_1}\cos{\theta_2} \\
		&= r_1^2 \dot{\theta}_1^2 + r_2^2 \dot{\theta}_2^2 + 2r_1 r_2 \dot{\theta}_1 \dot{\theta}_2 \cos{\left(\theta_1-\theta_2\right)}.
	\end{align*}
	
	Therefore the kinetic energy of the second bob is:
	
	\begin{align*}
		T_{\mathrm{bob}, 2} &= \dfrac{m_{2b} v_2^2}{2} \\
		&= \dfrac{m_{2b}}{2} \left(r_1^2 \dot{\theta}_1^2 + r_2^2 \dot{\theta}_2^2 + 2r_1 r_2 \dot{\theta}_1 \dot{\theta}_2 \cos{\left(\theta_1-\theta_2\right)}\right).
	\end{align*}
	
	Hence the total kinetic energy is:
	
	\begin{align*}
		T &= T_{\mathrm{rod}, 1} + T_{\mathrm{bob}, 1} + T_{\mathrm{rod}, 2} + T_{\mathrm{bob}, 2} \\
		&= \dfrac{m_{1r}r_1^2\dot{\theta}_1^2}{24} + \dfrac{m_{1b} r_1^2 \dot{\theta}_1^2}{2} + \dfrac{m_{2r} r_2^2\dot{\theta}_2^2}{24} + \dfrac{m_{2b}}{2} \left(r_1^2 \dot{\theta}_1^2 + r_2^2 \dot{\theta}_2^2 + 2r_1 r_2 \dot{\theta}_1 \dot{\theta}_2 \cos{\left(\theta_1-\theta_2\right)}\right) \\
		&= \dfrac{m_{1r} r_1^2 \dot{\theta}_1^2 + m_{2r}r_2^2 \dot{\theta}_2^2}{24} + \dfrac{m_{1b}+m_{2b}}{2}r_1^2 \dot{\theta}_1^2 + \dfrac{m_{2b}}{2} \left(r_2\dot{\theta}_2^2 + 2r_1 r_2 \dot{\theta}_1 \dot{\theta}_2 \cos{\left(\theta_1 - \theta_2\right)}\right).
	\end{align*}
	
	Next, we must calculate the potential energy. In this problem, there is only one source of potential energy---gravity. First, we have the gravitational potential energy of the first pendulum rod. For it, we will use the centre of mass of the pendulum for the y coordinate we use to calculate the gravitational potential energy. We will assume it has uniform mass, so its midpoint will be where its centre of mass is. 
	
	\begin{align*}
		V_{\mathrm{rod}, 1} &= m_{1r} gy_{\mathrm{rod}, 1} \\
		y_{\mathrm{rod}, 1} &= \dfrac{1}{2} y_1 \\
		&= \dfrac{r_1 \sin{\theta_1}}{2} \\
		V_{\mathrm{rod}, 1} &= \dfrac{m_{1r}gr_1 \sin{\theta_1}}{2}.
	\end{align*}
	
	As for the potential energy of the first pendulum bob:
	\begin{align*}
		V_{\mathrm{bob}, 1} &= m_{1b} gy_1 \\
		&= m_{1b}gr_1 \sin{\theta_1}.
	\end{align*}
	
	The potential energy of the second pendulum rod is:
	
	\begin{align*}
		V_{\mathrm{rod}, 2} &= m_{2r} gy_{\mathrm{rod}, 2} \\
		y_{\mathrm{rod}, 2} &= y_1 + \dfrac{y_2-y_1}{2}\\
		&= \dfrac{y_1+y_2}{2} \\
		&= r_1 \sin{\theta_1} + \dfrac{r_2\sin{\theta_2}}{2} \\
		V_{\mathrm{rod}, 2} &= m_{2r}g \left(r_1 \sin{\theta_1} + \dfrac{r_2\sin{\theta_2}}{2}\right).
	\end{align*}
	
	Finally, the potential energy of the second pendulum bob is:
	
	\begin{align*}
		V_{\mathrm{bob}, 2} &= m_{2b} gy_2 \\
		&= m_{2b} g \left(r_1 \sin{\theta_1} + r_2 \sin{\theta_2}\right).
	\end{align*}
	
	Therefore, the total potential energy is:
	\begin{align*}
		V &= V_{\mathrm{rod}, 1} + V_{\mathrm{rod}, 2} + V_{\mathrm{bob}, 1} + V_{\mathrm{bob}, 2} \\
		&= \dfrac{m_{1r}gr_1 \sin{\theta_1}}{2} + m_{2r}g \left(r_1 \sin{\theta_1} + \dfrac{r_2\sin{\theta_2}}{2}\right) + m_{1b}gr_1 \sin{\theta_1} + m_{2b} g \left(r_1 \sin{\theta_1} + r_2 \sin{\theta_2}\right).
	\end{align*}
	
	Hence the Lagrangian is:
	
	\begin{align*}
		\mathcal{L} &= T - V \\
		&= \dfrac{m_{1r} r_1^2 \dot{\theta}_1^2 + m_{2r}r_2^2 \dot{\theta}_2^2}{24} + \dfrac{m_{1b}+m_{2b}}{2}r_1^2 \dot{\theta}_1^2 + \dfrac{m_{2b}}{2} \left(r_2\dot{\theta}_2^2 + 2r_1 r_2 \dot{\theta}_1 \dot{\theta}_2 \cos{\left(\theta_1 - \theta_2\right)}\right) - \dfrac{m_{1r}gr_1 \sin{\theta_1}}{2} \\
		&-m_{2r}g \left(r_1 \sin{\theta_1} + \dfrac{r_2\sin{\theta_2}}{2}\right) - m_{1b}gr_1 \sin{\theta_1} - m_{2b} g \left(r_1 \sin{\theta_1} + r_2 \sin{\theta_2}\right).
	\end{align*}
	
	Hence:
	
	\begin{align*}
		p_{\theta_1} &= \dfrac{\partial \mathcal{L}}{\partial \dot{\theta}_1} \\
		&= \dfrac{m_{1r}r_1^2 \dot{\theta}_1}{12} + (m_{1b}+m_{2b}) r_1^2 \dot{\theta}_1 + m_{2b}r_1 r_2 \dot{\theta}_2 \cos{\left(\theta_1-\theta_2\right)} \\
		\dot{p}_{\theta_1} &= \dfrac{m_{1r} r_1^2 \ddot{\theta}_1}{12} + (m_{1b}+m_{2b})r_1^2 \ddot{\theta}_1 + m_{2b}r_1 r_2 \ddot{\theta}_2\cos{(\theta_1-\theta_2)} - m_{2b}r_1 r_2 \dot{\theta}_2\left(\dot{\theta}_1 - \dot{\theta}_2\right)\sin{(\theta_1-\theta_2)} \\
		F_{\theta_1} &= \dfrac{\partial \mathcal{L}}{\partial \theta_1} \\
		&= -m_{2b}r_1r_2\dot{\theta}_1\dot{\theta}_2 \sin{(\theta_1-\theta_2)} - \dfrac{m_{1r}gr_1 \cos{\theta_1}}{2} -m_{2r}gr_1 \cos{\theta_1} -m_{1b}gr_1 \cos{\theta_1} -m_{2b}gr_1 \cos{\theta_1}.
	\end{align*}
	
	As for the generalized dissipation force:
	
	\begin{align*}
		Q_{\theta_1} &= \vec{F}_{D, \mathrm{rod}, 1} \cdot \dfrac{\partial \vec{r}_{\mathrm{rod},1}}{\partial \theta_1} + \vec{F}_{D, \mathrm{bob}, 1} \cdot \dfrac{\partial \vec{r}_{\mathrm{bob},1}}{\partial \theta_1}+ \vec{F}_{D, \mathrm{rod}, 2} \cdot \dfrac{\partial \vec{r}_{rod, 2}}{\partial \theta_1} + \vec{F}_{D, \mathrm{bob}, 2} \cdot \dfrac{\partial \vec{r}_{\mathrm{bob}, 2}}{\partial \theta_1}
	\end{align*}
	
	\begin{align*}
		\vec{F}_{D,\mathrm{bob}, 1} &= -b_{\mathrm{bob}, 1}\vec{v}_1 - c_{\mathrm{bob}, 1}|\vec{v}_1|\vec{v}_1\\
		\therefore \vec{F}_{D, \mathrm{bob}, 1} \cdot \dfrac{\partial \vec{r}_{\mathrm{bob},1}}{\partial \theta_1}&= (-b_{\mathrm{bob}, 1} - c_{\mathrm{bob}, 1}r_1\dot{\theta}_1)\begin{bmatrix}
			-r_1\dot{\theta}_1 \sin{\theta_1} \\
			r_1\dot{\theta}_1 \cos{\theta_1}
			\end{bmatrix} \cdot \begin{bmatrix}
			-r_1 \sin{\theta_1} \\
			r_1 \cos{\theta_1}
		\end{bmatrix} \\
		&= -(b_{\mathrm{bob}, 1}+c_{\mathrm{bob}, 1}r_1\dot{\theta}_1) (r_1^2 \dot{\theta}_1 \sin^2{\theta_1}+r_1^2 \dot{\theta}_1 \cos^2{\theta_1}) \\
		&= -(b_{\mathrm{bob}, 1} + c_{\mathrm{bob}, 1} r_1 \dot{\theta}_1)r_1^2 \dot{\theta}_1 \\
		\vec{F}_{D,\mathrm{bob, 2}} &= -b_{\mathrm{bob}, 2}\vec{v}_2 - c_{\mathrm{bob}, 2}|\vec{v}_2|\vec{v}_2 \\
		&= -\left(b_{\mathrm{bob}, 2}+c_{\mathrm{bob}, 2}\sqrt{r_1^2 \dot{\theta}_1^2 + r_2^2 \dot{\theta}_2^2 +2r_1 r_2\dot{\theta}_1 \dot{\theta}_2 \cos{(\theta_1-\theta_2)}}\right) \begin{bmatrix}
			-r_1\dot{\theta}_1 \sin{\theta_1}-r_2\dot{\theta}_2 \sin{\theta_2} \\
			r_1\dot{\theta}_1 \cos{\theta_1}+r_2\dot{\theta}_2 \cos{\theta_2}
		\end{bmatrix} \\
		\vec{F}_{D, \mathrm{bob}, 2} \cdot \dfrac{\partial \vec{r}_{\mathrm{bob},2}}{\partial \theta_1} &= -\left(b_{\mathrm{bob}, 2}+c_{\mathrm{bob}, 2}\sqrt{r_1^2 \dot{\theta}_1^2 + r_2^2 \dot{\theta}_2^2 +2r_1 r_2\dot{\theta}_1 \dot{\theta}_2 \cos{(\theta_1-\theta_2)}}\right) \begin{bmatrix}
			-r_1\dot{\theta}_1 \sin{\theta_1}-r_2\dot{\theta}_2 \sin{\theta_2} \\
			r_1\dot{\theta}_1 \cos{\theta_1}+r_2\dot{\theta}_2 \cos{\theta_2}
		\end{bmatrix} \\&\cdot \begin{bmatrix}
		-r_1 \sin{\theta_1} \\
		r_1 \cos{\theta_1}
		\end{bmatrix}\\
		&= -\left(b_{\mathrm{bob}, 2}+c_{\mathrm{bob}, 2}\sqrt{r_1^2 \dot{\theta}_1^2 + r_2^2 \dot{\theta}_2^2 +2r_1 r_2\dot{\theta}_1 \dot{\theta}_2 \cos{(\theta_1-\theta_2)}}\right)(r_1^2 \dot{\theta}_1 \sin^2{\theta_1}+r_1r_2 \dot{\theta}_2 \sin{\theta_1}\sin{\theta_2} \\
		&+r_1^2 \dot{\theta_1}\cos^2{\theta_1}+r_1r_2\dot{\theta}_2 \cos{\theta_1}\cos{\theta_2}) \\
		&= -\left(b_{\mathrm{bob}, 2}+c_{\mathrm{bob}, 2}\sqrt{r_1^2 \dot{\theta}_1^2 + r_2^2 \dot{\theta}_2^2 +2r_1 r_2\dot{\theta}_1 \dot{\theta}_2 \cos{(\theta_1-\theta_2)}}\right)(r_1^2 \dot{\theta}_1 + r_1r_2 \dot{\theta}_2 \cos{(\theta_1-\theta_2)}).
	\end{align*}
	
	Next we will calculate the dissipation force on the first rod. We will use a centre of mass approximation (as otherwise we would likely have to integrate over the rod, which would drastically complicate the calculation):
	
	\begin{align*}
		\vec{F}_{D,\mathrm{rod}, 1} &= -\left(b_{\mathrm{rod}, 1} + c_{\mathrm{rod}, 1}|\vec{v}_{\mathrm{rod}, 1}|\right)\vec{v}_{\mathrm{rod}, 1} \\
		\vec{v}_{\mathrm{rod}, 1} &= \begin{bmatrix}
			\dfrac{-r_1 \dot{\theta}_1 \sin{\theta_1}}{2} \\
			\dfrac{r_1 \dot{\theta}_1 \cos{\theta_1}}{2}
		\end{bmatrix} \\
		&= \dfrac{r_1 \dot{\theta}_1}{2} \begin{bmatrix}
			-\sin{\theta_1} \\
			\cos{\theta_1}
		\end{bmatrix} \\
		\therefore |\vec{v}_{\mathrm{rod}, 1}| &= \dfrac{r_1 \dot{\theta}_1}{2} \\
		\vec{F}_{D,\mathrm{rod}, 1} &= -\left(b_{\mathrm{rod}, 1} + \dfrac{c_{\mathrm{rod}, 1}r_1 \dot{\theta}_1}{2}\right)\dfrac{r_1 \dot{\theta}_1}{2} \begin{bmatrix}
			-\sin{\theta_1} \\
			\cos{\theta_1}
		\end{bmatrix} \\
		\dfrac{\partial \vec{r}_{\mathrm{rod}, 1}}{\partial \theta_1} &= \dfrac{r_1}{2}\begin{bmatrix}
			-\sin{\theta_1} \\
			\cos{\theta_1}
		\end{bmatrix} \\
		\vec{F}_{D,\mathrm{rod}, 1} \cdot \dfrac{\partial \vec{r}_{\mathrm{rod}, 1}}{\partial \theta_1} &= -\left(b_{\mathrm{rod}, 1} + \dfrac{c_{\mathrm{rod}, 1}r_1 \dot{\theta}_1}{2}\right)\dfrac{r_1 \dot{\theta}_1}{2} \begin{bmatrix}
			-\sin{\theta_1} \\
			\cos{\theta_1}
		\end{bmatrix} \cdot \dfrac{r_1}{2}\begin{bmatrix}
			-\sin{\theta_1} \\
			\cos{\theta_1}
		\end{bmatrix} \\
		&= -\left(b_{\mathrm{rod}, 1} + \dfrac{c_{\mathrm{rod}, 1}r_1 \dot{\theta}_1}{2}\right) \dfrac{r_1^2 \dot{\theta}_1}{4} \\
	\end{align*}
	\begin{align*}
		\vec{F}_{D,\mathrm{rod}, 2} &= -\left(b_{\mathrm{rod}, 2} + c_{\mathrm{rod}, 2}|\vec{v}_{\mathrm{rod}, 2}|\right)\vec{v}_{\mathrm{rod}, 2} \\
		\vec{v}_{\mathrm{rod}, 2} &= \begin{bmatrix}
			-r_1 \dot{\theta}_1\sin{\theta_1} - \dfrac{r_2\dot{\theta}_2\sin{\theta_2}}{2}\\
			r_1 \dot{\theta}_1 \cos{\theta_1} + \dfrac{r_2\dot{\theta}_2\cos{\theta_2}}{2}
		\end{bmatrix} \\
		|\vec{v}_{\mathrm{rod}, 2}|^2 &= \left(	-r_1 \dot{\theta}_1\sin{\theta_1} - \dfrac{r_2\dot{\theta}_2\sin{\theta_2}}{2}\right)^2 + \left(r_1 \dot{\theta}_1 \cos{\theta_1} + \dfrac{r_2\dot{\theta}_2\cos{\theta_2}}{2}\right)^2 \\
		&= r_1^2 \dot{\theta}_1^2 \sin^2{\theta_1} + r_1 r_2 \dot{\theta}_1 \dot{\theta}_2\sin{\theta_1}\sin{\theta_2} + \dfrac{r_2^2 \dot{\theta}_2^2 \sin^2{\theta_2}}{4} + r_1^2 \dot{\theta}_1^2 \cos^2{\theta_1} + r_1r_2 \dot{\theta}_1 \dot{\theta}_2 \cos{\theta_1}\cos{\theta_2}\\
		&+ \dfrac{r_2^2 \dot{\theta}_2^2 \cos^2{\theta_2}}{4} \\
		&= r_1^2 \dot{\theta}_1^2 + \dfrac{r_2^2 \dot{\theta}_2^2}{4} + r_1 r_2 \dot{\theta}_1 \dot{\theta}_2 \cos{(\theta_1 -\theta_2)} \\
		|\vec{v}_{\mathrm{rod}, 2}| &= \sqrt{r_1^2 \dot{\theta}_1^2 + \dfrac{r_2^2 \dot{\theta}_2^2}{4} + r_1 r_2 \dot{\theta}_1 \dot{\theta}_2 \cos{(\theta_1 -\theta_2)}} \\
		\vec{F}_{D,\mathrm{rod}, 2} &= -\left(b_{\mathrm{rod}, 2} + c_{\mathrm{rod}, 2}\sqrt{r_1^2 \dot{\theta}_1^2 + \dfrac{r_2^2 \dot{\theta}_2^2}{4} + r_1 r_2 \dot{\theta}_1 \dot{\theta}_2 \cos{(\theta_1 -\theta_2)}}\right)\begin{bmatrix}
			-r_1 \dot{\theta}_1\sin{\theta_1} - \dfrac{r_2\dot{\theta}_2\sin{\theta_2}}{2}\\
			r_1 \dot{\theta}_1 \cos{\theta_1} + \dfrac{r_2\dot{\theta}_2\cos{\theta_2}}{2}
		\end{bmatrix} \\
		\dfrac{\partial \vec{r}_{\mathrm{rod}, 2}}{\partial \theta_1} &= \begin{bmatrix}
			-r_1 \sin{\theta_1} \\
			r_1 \cos{\theta_1}
		\end{bmatrix}\\
		\vec{F}_{D,\mathrm{rod}, 2}\cdot \dfrac{\partial \vec{r}_{\mathrm{rod}, 2}}{\partial \theta_1} &= -\left(b_{\mathrm{rod}, 2} + c_{\mathrm{rod}, 2}\sqrt{r_1^2 \dot{\theta}_1^2 + \dfrac{r_2^2 \dot{\theta}_2^2}{4} + r_1 r_2 \dot{\theta}_1 \dot{\theta}_2 \cos{(\theta_1 -\theta_2)}}\right) (r_1^2 \dot{\theta}_1 \sin^2{\theta_1} + \dfrac{r_1 r_2 \dot{\theta}_2 \sin{\theta_1}\sin{\theta_2}}{2} \\
		&+r_1^2 \dot{\theta}_1\cos^2{\theta_1} + \dfrac{r_1r_2 \dot{\theta}_2 \cos{\theta_1}\cos{\theta_2}}{2}) \\
		&= -\left(b_{\mathrm{rod}, 2} + c_{\mathrm{rod}, 2}\sqrt{r_1^2 \dot{\theta}_1^2 + \dfrac{r_2^2 \dot{\theta}_2^2}{4} + r_1 r_2 \dot{\theta}_1 \dot{\theta}_2 \cos{(\theta_1 -\theta_2)}}\right)\left(r_1^2 \dot{\theta}_1^2 + \dfrac{r_1 r_2\dot{\theta}_2 \cos{\left(\theta_1 - \theta_2\right)}}{2}\right).
	\end{align*}
	
	Hence $Q_{\theta_1}$ is:
	\begin{align*}
		Q_{\theta_1} &= -(b_{\mathrm{bob}, 1} + c_{\mathrm{bob}, 1} r_1 \dot{\theta}_1)r_1^2 \dot{\theta}_1 -\left(b_{\mathrm{bob}, 2}+c_{\mathrm{bob}, 2}\sqrt{r_1^2 \dot{\theta}_1^2 + r_2^2 \dot{\theta}_2^2 +2r_1 r_2\dot{\theta}_1 \dot{\theta}_2 \cos{(\theta_1-\theta_2)}}\right)(r_1^2 \dot{\theta}_1 \\
		&+ r_1r_2 \dot{\theta}_2 \cos{(\theta_1-\theta_2)}) -\left(b_{\mathrm{rod}, 1} + \dfrac{c_{\mathrm{rod}, 1}r_1 \dot{\theta}_1}{2}\right) \dfrac{r_1^2 \dot{\theta}_1}{4} \\
		& -\left(b_{\mathrm{rod}, 2} + c_{\mathrm{rod}, 2}\sqrt{r_1^2 \dot{\theta}_1^2 + \dfrac{r_2^2 \dot{\theta}_2^2}{4} + r_1 r_2 \dot{\theta}_1 \dot{\theta}_2 \cos{(\theta_1 -\theta_2)}}\right)\left(r_1^2 \dot{\theta}_1^2 + \dfrac{r_1 r_2\dot{\theta}_2 \cos{\left(\theta_1 - \theta_2\right)}}{2}\right).
	\end{align*}
	
	The equation of motion for $\theta_1$ is therefore:
	
	\begin{align*}
		\dot{p}_{\theta_1} - F_{\theta_1} &= Q_{\theta_1}
	\end{align*}
	
	\begin{align*}
		&\dfrac{m_{1r} r_1^2 \ddot{\theta}_1}{12} + (m_{1b}+m_{2b})r_1^2 \ddot{\theta}_1 + m_{2b}r_1 r_2 \ddot{\theta}_2\cos{(\theta_1-\theta_2)} - m_{2b}r_1 r_2 \dot{\theta}_2\left(\dot{\theta}_1 - \dot{\theta}_2\right)\sin{(\theta_1-\theta_2)} \\
		&- (-m_{2b}r_1r_2\dot{\theta}_1\dot{\theta}_2 \sin{(\theta_1-\theta_2)} - \dfrac{m_{1r}gr_1 \cos{\theta_1}}{2} -m_{2r}gr_1 \cos{\theta_1} -m_{1b}gr_1 \cos{\theta_1} -m_{2b}gr_1 \cos{\theta_1}) = -(b_{\mathrm{bob}, 1} \\
		&+ c_{\mathrm{bob}, 1} r_1 \dot{\theta}_1)r_1^2 \dot{\theta}_1 -\left(b_{\mathrm{bob}, 2}+c_{\mathrm{bob}, 2}\sqrt{r_1^2 \dot{\theta}_1^2 + r_2^2 \dot{\theta}_2^2 +2r_1 r_2\dot{\theta}_1 \dot{\theta}_2 \cos{(\theta_1-\theta_2)}}\right)(r_1^2 \dot{\theta}_1 \\
		&+ r_1r_2 \dot{\theta}_2 \cos{(\theta_1-\theta_2)}) -\left(b_{\mathrm{rod}, 1} + \dfrac{c_{\mathrm{rod}, 1}r_1 \dot{\theta}_1}{2}\right) \dfrac{r_1^2 \dot{\theta}_1}{4} \\
		& -\left(b_{\mathrm{rod}, 2} + c_{\mathrm{rod}, 2}\sqrt{r_1^2 \dot{\theta}_1^2 + \dfrac{r_2^2 \dot{\theta}_2^2}{4} + r_1 r_2 \dot{\theta}_1 \dot{\theta}_2 \cos{(\theta_1 -\theta_2)}}\right)\left(r_1^2 \dot{\theta}_1^2 + \dfrac{r_1 r_2\dot{\theta}_2 \cos{\left(\theta_1 - \theta_2\right)}}{2}\right) \\
		&\dfrac{m_{1r} r_1^2 \ddot{\theta}_1}{12} + (m_{1b}+m_{2b})r_1^2 \ddot{\theta}_1 + m_{2b}r_1 r_2 \ddot{\theta}_2\cos{(\theta_1-\theta_2)} + m_{2b}r_1 r_2 \dot{\theta}_2^2\sin{(\theta_1-\theta_2)} \\
		&+ \dfrac{m_{1r}gr_1 \cos{\theta_1}}{2} +m_{2r}gr_1 \cos{\theta_1} +m_{1b}gr_1 \cos{\theta_1} +m_{2b}gr_1 \cos{\theta_1} = -(b_{\mathrm{bob}, 1} + c_{\mathrm{bob}, 1} r_1 \dot{\theta}_1)r_1^2 \dot{\theta}_1 \\
		&-\left(b_{\mathrm{bob}, 2}+c_{\mathrm{bob}, 2}\sqrt{r_1^2 \dot{\theta}_1^2 + r_2^2 \dot{\theta}_2^2 +2r_1 r_2\dot{\theta}_1 \dot{\theta}_2 \cos{(\theta_1-\theta_2)}}\right)(r_1^2 \dot{\theta}_1 + r_1r_2 \dot{\theta}_2 \cos{(\theta_1-\theta_2)}) -\left(b_{\mathrm{rod}, 1} + \dfrac{c_{\mathrm{rod}, 1}r_1 \dot{\theta}_1}{2}\right) \dfrac{r_1^2 \dot{\theta}_1}{4} \\
		& -\left(b_{\mathrm{rod}, 2} + c_{\mathrm{rod}, 2}\sqrt{r_1^2 \dot{\theta}_1^2 + \dfrac{r_2^2 \dot{\theta}_2^2}{4} + r_1 r_2 \dot{\theta}_1 \dot{\theta}_2 \cos{(\theta_1 -\theta_2)}}\right)\left(r_1^2 \dot{\theta}_1^2 + \dfrac{r_1 r_2\dot{\theta}_2 \cos{\left(\theta_1 - \theta_2\right)}}{2}\right)
	\end{align*}
\end{document}